%%%%%%%%%%%%%%%%%%%%%%%%%%%%%%%%%%%%%%%%%
% Medium Length Professional CV
% LaTeX Template
% Version 2.0 (8/5/13)
%
% This template has been downloaded from:
% http://www.LaTeXTemplates.com
%
% Original author:
% Rishi Shah 
%
% Important note:
% This template requires the resume.cls file to be in the same directory as the
% .tex file. The resume.cls file provides the resume style used for structuring the
% document.
%
%%%%%%%%%%%%%%%%%%%%%%%%%%%%%%%%%%%%%%%%%

%----------------------------------------------------------------------------------------
%	PACKAGES AND OTHER DOCUMENT CONFIGURATIONS
%----------------------------------------------------------------------------------------

\documentclass{resume} % Use the custom resume.cls style

\usepackage{hyperref}
\usepackage[left=0.75in,top=0.6in,right=0.75in,bottom=0.6in]{geometry} % Document margins
\newcommand{\tab}[1]{\hspace{.2667\textwidth}\rlap{#1}}
\newcommand{\itab}[1]{\hspace{0em}\rlap{#1}}
\name{Anton Zavyalov} % Your name
%\address{123 Pleasant Lane \\ City, State 12345} % Your secondary addess (optional)
\address{+7 960 956 61 14 \\ megadeathlightsaber@gmail.com} % Your phone number and email

\hypersetup{
  colorlinks=true,
  urlcolor=blue
}

\begin{document}

%----------------------------------------------------------------------------------------
%	EDUCATION SECTION
%----------------------------------------------------------------------------------------

\begin{rSection}{Education}

{\bf Bachelor of Software Engineering} \\ Polzunov Altai State Technical University, Barnaul \hfill {September 2017 - July 2021}

Faculty of Information Technologies
\\Speciality 09.03.04 --- Software Engineering

{\bf Bachelor thesis} \\ Design of visual functional language with recursion optimization capabilities,
\\ supervised by associate professor Sergey Mikhailovich Starlovetov \\
%Minor in Linguistics \smallskip \\
%Member of Eta Kappa Nu \\
%Member of Upsilon Pi Epsilon \\


\end{rSection}

%\begin{rSection}{Programming background}
%I have been programming since I was 11. At first I was interested in modding and game development, so my first programming languages were Lua, C\# and Haxe (Google it, it's awesome!).

%From the first year in university I am interested in the theory of programming languages and functional programming. I am familiar with such languages as Scheme, Haskell, OCaml and others. Obviously, my main interests are system programming and programming language design. My experience in programming language development includes a simple LISP interpreter with nested scope and higher order functions; an interpreter of a C-like language  and LLVM-powered compiler for it.

%My interests do not contradict the fact that I am open to any interesting job offers. A big bonus if the offer is related to functional programming and programming language theory. ;)
%\end{rSection}
%--------------------------------------------------------------------------------
%    Projects And Seminars
%-----------------------------------------------------------------------------------------------
\begin{rSection}{Personal Projects}
\item \begin{rSubsection}{Flovver}{January 2021 - June 2021}{}{}
Visual functional programming language and rapid application development environment. Subject of my bachelor thesis in Polzunov Altai State Technical University.

Thesis includes the following points:
\begin{itemize}
    \item Analysis of technologies and algorithms of recursion optimization in compilers.
    \item Design of visual functional programming language.
    \item Design and implementation of optimizing compiler and IDE for proposed language.
\end{itemize}

All the works were done by me with the use of \textit{Scala} and \textit{Svelte}.

URL: \url{https://github.com/flovver}
\end{rSubsection}
    
\item \begin{rSubsection}{evply --- Graphical LISP Interpreter}{May 2019}{}{}
The goal of this project is to make symbolic computations more symbolic. You have a workspace where you can place and move primitives: numbers, lines, images, functions, etc. You select these primitives, click the button on the toolbar --- and the selected objects are grouped into a symbolic expression.

A prototype of \textit{evply} was developed for Hackathon Barnaul 2019.

As a part of team, I designed the language and developed an interpreter for it.

URL: \url{https://github.com/andiogenes/evply}

Slides: \url{https://raw.githubusercontent.com/andiogenes/evply/media/slides.pdf}
\end{rSubsection}

\item \begin{rSubsection}{Daria}{June 2020}{}{}
Toy programming language with naive pattern matching design and implementation.

URL: \url{https://github.com/andiogenes/daria}
\end{rSubsection}

\item \begin{rSubsection}{The Carrot programming language}{March 2021}{}{}
Experimental concatenative programming language with interpreter written in OCaml.

URL: \url{https://github.com/andiogenes/carrot}
\end{rSubsection}
\end{rSection}
%----------------------------------------------------------------------------------------
%	TECHNICAL STRENGTHS SECTION
%----------------------------------------------------------------------------------------

\clearpage

\begin{rSection}{Computer Skills}

\begin{tabular}{ @{} >{\bfseries}l @{\hspace{6ex}} l }
Programming Languages \ & Scala, Kotlin, Java, Go, C++, Python \\
Software \& Tools & Linux, Git, Docker \\
\end{tabular}

\end{rSection}

%----------------------------------------------------------------------------------------
%	WORK EXPERIENCE SECTION
%----------------------------------------------------------------------------------------

\begin{rSection}{Work Experience}

\item \begin{rSubsection}{Stackeer.io}{December 2019 - April 2020}{Backend Developer}{Barnaul}
I worked on UnReview --- ML-based recommendation system intended to select reviewers for PRs 
from among the contributors.

Working with Stackeer.io, I made the following things:
\begin{itemize}
    \item Implement a number of microservices in Go. Microservices are connected by REST and gRPC and
are managed by an orchestration system.
    \item Create a command line tool to fetch data about repositories with GitHub GraphQLv4.
    \item Refactor a legacy library consisting of one class with a large number of methods and fields. To do this, I wrote an ad-hoc tool that traverses an AST and emits a call graph. Next, this graph is rendered through GraphViz and I can see how to perform refactoring.
\end{itemize}
\end{rSubsection}

\item \begin{rSubsection}{Manpower for Huawei}{July 2020 - September 2021}{Assistant Engineer}{Novosibirsk}
Work on JVM and compiler related stuff at Excelsior Team.
\end{rSubsection}

\item \begin{rSubsection}{Huawei}{September 2021 - Present}{Engineer}{Novosibirsk}
Keep working on JVM and compiler related stuff. :)
\end{rSubsection}

\end{rSection}


%	EXAMPLE SECTION
%----------------------------------------------------------------------------------------

\begin{rSection}{Academic Achievements} 
\begin{itemize}
\item {Member of the competitive programming team in Polzunov Altai State Technical University in 2018-2020.

Participated in ICPC 2019-2020, NERC --- Northern Eurasia Finals. Siberian and Far-Eastern Site. Our team took 162th place out of 250 in the global standings (16th out of 49 in the regional standings).}
\item Took 2nd place in the 18th All-Russian Scientific and Technical Conference of Students, Postgraduates and Young Scientists "Nauka I Molodezh" in the "Software Engineering" subsection of the "Information Technologies" section. 
\end{itemize}
\end{rSection}

\begin{rSection}{Publications}
\begin{itemize}
    \item Zavyalov, A. Design of visual functional language with recursion description capabilities. / "Nauka I Molodezh": Materials of 18th All-Russian Scientific and Technical Conference of Students, Postgraduates and
    Yong Scientists
\end{itemize}
\end{rSection}

\newpage

%----------------------------------------------------------------------------------------
% Extra Curricular
%----------------------------------------------------------------------------------------
\begin{rSection}{Extra-Cirrucular} \itemsep -3pt
\item Received a special award (Bitcoin-shaped chocolate bar) at Hackathon Barnaul 2019.
\item Played bass in a post-metal band called \href{https://sunhurt.bandcamp.com/}{Sunhurt} in 2019.
\item Linux ricing enthusiast.

\end{rSection}

\begin{rSection}{Personal Traits}
\item Highly motivated and eager to learn new things.
\item Ability to work as an individual as well as in group.
\end{rSection}
\end{document}
